\documentclass[11pt]{article}

% Change "review" to "final" to generate the final (sometimes called camera-ready) version.
% Change to "preprint" to generate a non-anonymous version with page numbers.
\usepackage[review]{acl}

% Standard package includes
\usepackage{times}
\usepackage{latexsym}

% For proper rendering and hyphenation of words containing Latin characters (including in bib files)
\usepackage[T1]{fontenc}
% This assumes your files are encoded as UTF8
\usepackage[utf8]{inputenc}

% This is not strictly necessary, and may be commented out,
% but it will improve the layout of the manuscript,
% and will typically save some space.
\usepackage{microtype}

% This is also not strictly necessary, and may be commented out.
% However, it will improve the aesthetics of text in
% the typewriter font.
\usepackage{inconsolata}
\usepackage{booktabs}
\usepackage{adjustbox}
\usepackage{tabularx}
\usepackage{siunitx}
\sisetup{retain-explicit-plus=true}
\usepackage{float}
\usepackage[section]{placeins}
% pgfplots not needed after switching to tables

%Including images in your LaTeX document requires adding
%additional package(s)
\usepackage{graphicx}
\usepackage{textgreek}
\usepackage{textcomp}
\usepackage{amssymb}
\usepackage{amsmath}
\usepackage[colorinlistoftodos,prependcaption,textsize=tiny]{todonotes}

\title{Personality Manipulation in Large Language Models: A Systematic Probe into Behavioral Representation and Control Mechanisms}

\author{Author Name 1 \\
  Institution \\
  \texttt{email@domain.com} \\\And
  Author Name 2 \\
  Institution \\
  \texttt{email@domain.com} \\}

\begin{document}
\maketitle
\begin{abstract}
While personality design has become common practice in large language models through prompt engineering and fine-tuning, the underlying mechanisms and downstream effects remain poorly understood. This paper uses systematic personality manipulation as a probe to understand how behavioral traits are encoded and controlled in LLMs through three complementary approaches: in-context learning (ICL), parameter-efficient fine-tuning (PEFT), and activation-based steering vectors. We develop a controlled experimental platform to validate Big Five personality trait induction and measure its impact on bias and task performance. Our study evaluates ICL, LoRA-style PEFT on Gemma-2-2B-IT and LLaMA-3-8B-Instruct, and novel activation steering vectors derived from layer-wise activation differences. Results demonstrate that ICL generates large immediate trait shifts, PEFT produces more stable personality changes, and activation steering achieves competitive effectiveness with lightweight computational requirements. Downstream evaluation on BBQ, MMLU, and GAIA benchmarks reveals distinct trade-offs between personality control strength and task performance across methods. Our manipulation experiments reveal distinct mechanisms underlying personality control and establish activation-based steering as an interpretable window into personality representation space, providing both practical guidance and fundamental insights into behavioral control mechanisms in neural language models.
\end{abstract}

\section{Introduction}

Designing and controlling personality in large language models (LLMs) is increasingly common, yet the trade-offs between personality control and task capability are poorly quantified. Prior work spans prompting-based conditioning, parameter-efficient fine-tuning (PEFT; e.g., LoRA), and activation-based steering of internal representations, alongside evaluation frameworks for knowledge (MMLU), reasoning (GAIA), and bias (BBQ). However, consistent comparisons across methods and traits remain limited.

We study three families of approaches for Big Five personality manipulation: prompting, PEFT (LoRA adapters), and activation steering. To ensure fair comparison despite run-to-run baseline differences, we adopt a relative change (\(\Delta\)) analysis: all effects are measured within each method's own baseline run. We complement benchmark outcomes with an independent alignment validation task that directly measures how strongly a target personality is expressed.

Our contributions are threefold: (1) a unified, \(\Delta\)-based evaluation that isolates method effects independent of absolute baselines; (2) a cross-method, cross-trait analysis tying \(\Delta\) capability changes to independent alignment strength; and (3) practical guidance on method selection under capability constraints, with implementation details and extended results in the appendices.
\section{Methods}

\textbf{Setup.} We evaluate personality manipulation on Gemma-2-2B-IT and LLaMA-3-8B-Instruct across MMLU (strategic subjects; \(N=50\) per subject), GAIA 2023 Level 1 (\(N=53\)), and BBQ filtered to ambiguous questions using official metadata (Appendix~\ref{app:benchmarks}). We target Big Five traits and report effects \emph{within} each method's run using a relative change (\(\Delta\)) analysis (Appendix~\ref{app:experimental-design}).

\textbf{Manipulation methods.} (1) \emph{In-context learning (ICL)}: full-context persona prompts with exemplars drawn from the Holistic AI personality dataset; this approach probes surface-level accessibility of personality representations without modifying model parameters (Appendix~\ref{app:icl}). (2) \emph{PEFT}: trait-specific LoRA adapters trained on contrastive personality pairs produced from the same dataset; this reveals which model components are critical for stable personality encoding through targeted parameter modifications (Appendix~\ref{app:peft}). (3) \emph{Activation steering}: add a calibrated vector at a target transformer layer's post-attention layer norm; Gemma vectors use trait contrast; this provides direct manipulation of representational depths, revealing layer-specific personality encoding mechanisms (Appendix~\ref{app:activation-steering}).

\textbf{Generation and scoring.} Stage 1 generates responses per benchmark and trait (plus Baseline). Stage 2 scores: MMLU/GAIA by accuracy; BBQ by \(S_{AMB}\) only (we ignore \(S_{DIS}\)). Final-answer extraction uses an Azure GPT judge. Personality alignment on responses is measured via a public classifier, and we additionally run a dedicated alignment task (reported separately) to validate trait expression strength.

\textbf{Primary metrics.} For MMLU and GAIA we report \(\Delta\) Accuracy relative to the method's Baseline; for BBQ we report \(\Delta S_{AMB}\). We use dedicated alignment scores (manipulated vs baseline) as the primary alignment metric and treat benchmark-derived alignment as secondary context.
\section{Results}

\textbf{Framing.} We report \(\Delta\) from each method's own Baseline within its run: MMLU uses \(\Delta\) Accuracy\_Avg, GAIA uses \(\Delta\) Accuracy, and BBQ uses \(\Delta S_{AMB}\). We ignore \(S_{DIS}\). Alignment is validated with a dedicated task (independent of benchmarks).

\textbf{Main findings.}


\textbf{Gemma-2, MMLU:} Manipulation experiments reveal distinct representational pathways: ICL exhibits \emph{modest negative} \(\Delta\) across traits, suggesting surface-level conditioning; Steering exhibits \emph{large negative} \(\Delta\) for several traits, indicating that deep representational intervention disrupts critical pathways; PEFT shows trait-dependent \(\Delta\), often negative for some traits and small for others.


\textbf{Gemma-2, GAIA:} ICL shows \emph{small positive} \(\Delta\) on average; PEFT and Steering generally show small negative \(\Delta\).


\textbf{LLaMA-3, MMLU/GAIA:} ICL and PEFT both yield small within-run \(\Delta\); we avoid cross-run absolute comparisons.


\textbf{BBQ (Gemma-2 \/ LLaMA-3):} \(\Delta S_{AMB}\) is trait- and method-dependent: ICL effects are generally small, while Steering and PEFT can induce large negative shifts for some traits on Gemma-2. We do not use \(S_{DIS}\).

\textbf{Alignment validation (independent task).} Systematic manipulation uncovers differential trait accessibility: ICL and PEFT achieve strong trait alignment across models (e.g., Gemma extraversion: 1.00 ICL, 0.96 PEFT; LLaMA neuroticism: 1.00 for both). Steering shows statistically significant alignment across assessed traits on Gemma-2. Agreeableness emerges as the most challenging trait for ICL, revealing potential representational complexity differences across personality dimensions.

% Detailed delta tables are provided in Appendix~\ref{app:downstream-analysis} for MMLU (per subject), GAIA, and BBQ.

\textbf{Notes.} Absolute baselines vary across runs and are not compared. Detailed per-subject MMLU deltas, alignment tables, training settings, and steering calibration details are provided in the appendices.
\section{Discussion}

\textbf{Trade-offs.} ICL delivers strong alignment with small \(\Delta\) capability changes; PEFT maximizes alignment but incurs large negative \(\Delta\) (notably Gemma-2 MMLU/GAIA); steering offers moderate alignment with trait-dependent \(\Delta\), improving with careful vector construction (e.g., purified openness). Detailed comparative analysis is in Appendix~\ref{app:comparative-analysis}.

\textbf{Manipulation Reveals Personality Encoding.} Three key insights emerge: (1) ICL effectiveness with minimal disruption shows personality traits are accessible through surface-level conditioning without affecting core representations; (2) PEFT's strong alignment but capability costs show personality can be deeply encoded in parameters at the expense of general capabilities, suggesting shared resources; (3) Activation steering's layer-specific effectiveness (typically Layer 15) maps personality traits to specific depths, showing encoding consolidates at intermediate transformer layers.

\textbf{Manipulation as Interpretability Tool.} Different intervention approaches reveal complementary aspects of personality representation. The \(\Delta\)-analysis framework isolates method-specific effects and shows which pathways each technique accesses. This establishes manipulation-based interventions as a viable interpretability methodology for understanding behavioral control in neural language models.

\textbf{Trait/model dependence.} Agreeableness is hardest to align via ICL; openness benefits from vector composition; LLaMA-3 shows different baselines across runs, so we interpret only within-run \(\Delta\). Complete trait-wise results are in Appendix~\ref{app:trait-results}.

\textbf{Practical guidance.} When preserving capability is critical, prefer ICL; use steering when lightweight, runtime control is needed and calibration is feasible; reserve PEFT for settings where strong, stable personality alignment outweighs capability costs.

\textbf{Limitations.} We avoid cross-run baseline comparisons by design (\(\Delta\)-only). We ignore \(S_{DIS}\) for BBQ and rely on ambiguous bias score \(S_{AMB}\). Alignment is validated independently; detailed settings and analyses are in appendices. Extended discussion of limitations, ethical considerations, and future work is in Appendix~\ref{app:discussion-extended}.

% Appendices
\appendix
\section{Background and Related Work}
\label{app:background}
\noindent\textbf{Evaluation frame.} Throughout, we report within-run relative changes (\(\Delta\)) for fairness across methods with differing absolute baselines, and validate personality alignment using both benchmark classification and a dedicated alignment task.

\noindent\textbf{Background on LLM personality.} Prior work documents baseline personality expression and surveys of methods and measures \citep{serapio-garcia-etal-2023-personality-traits-llms,jiang-2024-personallm-naacl,wen-2024-llm-personality-survey}. Direct application of human psychometrics to LLMs shows instability and validity concerns, motivating behavioral validation \citep{gupta-2024-psychometric-validity-llms,song-2023-stability-llm-personality}.

\noindent\textbf{Method taxonomy.} We situate in-context learning (ICL) \citep{mao-2023-editing-personality}, parameter-efficient fine-tuning (LoRA/QLoRA) \citep{hu-etal-2022-lora,dettmers-2023-qlora}, and activation engineering/steering \citep{turner-etal-2023-activation-steering,panickssery-2024-contrastive-activation-addition,chen-2025-persona-vectors} as complementary approaches.

\noindent\textbf{Safety and bias context.} We evaluate social bias using BBQ \citep{parrish-etal-2022-bbq}, with related literature on toxicity and safety effects of personas \citep{gehman-2020-realtoxicityprompts,zhang-2024-better-angels,wang-2025-personality-bias-toxicity,durmus-2024-evaluating-feature-steering}.

Personality conditioning can modulate toxic or biased tendencies in LLM outputs; we therefore quantify bias effects alongside capability deltas and validate that induced personas align behaviorally \citep{gehman-2020-realtoxicityprompts,wang-2025-personality-bias-toxicity}.

\noindent\textbf{Mechanistic perspective.} Our use of activation-space interventions connects to mechanistic interpretability \citep{olah-2020-circuits,bricken-2023-monosemanticity,elhage-2022-superposition,rai-2024-mechinterp-review}.

\subsection{Personality in Language Models}

The computational modeling of personality in language systems has evolved from early rule-based approaches to sophisticated neural architectures. \citet{mairesse-walker-2007-personage} established foundational work in personality-driven text generation, demonstrating how linguistic features correlate with Big Five personality traits. Recent work has extended these concepts to large language models, with \citet{jiang-etal-2023-personallm} showing that LLMs can exhibit consistent personality-like behaviors when properly conditioned.

The Big Five personality model (Openness, Conscientiousness, Extraversion, Agreeableness, Neuroticism) has emerged as the dominant framework for computational personality research due to its empirical validation and cross-cultural applicability \citep{costa-mccrae-1992-big5}. \citet{karra-etal-2022-ai-personality} demonstrated that LLMs can be assessed using established personality questionnaires, while \citet{huang-etal-2023-chatgpt-personality} revealed that models like ChatGPT exhibit detectable personality patterns even without explicit conditioning.

\subsection{In-Context Learning for Behavior Control}

In-context learning has become a primary method for controlling LLM behavior without modifying model parameters. \citet{wei-etal-2022-chain-of-thought} showed how carefully designed prompts can significantly alter reasoning patterns, while \citet{liu-etal-2023-pre-train-prompt-tune} demonstrated the effectiveness of prompt-based conditioning for various behavioral modifications.

Specifically for personality conditioning, \citet{wang-etal-2023-roleplay-prompting} explored how role-playing prompts can induce consistent personality traits, finding that detailed character descriptions lead to more stable personality expressions. \citet{li-etal-2023-personality-prompting} systematically evaluated different ICL strategies for Big Five trait induction, establishing baseline effectiveness measures that inform our experimental design.

\subsection{Parameter-Efficient Fine-tuning}

Parameter-efficient fine-tuning methods have gained prominence as alternatives to full model fine-tuning, offering computational efficiency while maintaining performance. Low-Rank Adaptation (LoRA) \citep{hu-etal-2022-lora} has become particularly popular, enabling targeted parameter updates through low-rank matrix decompositions.

\citet{zhang-etal-2023-peft-personality} were among the first to apply PEFT methods specifically for personality conditioning, demonstrating that LoRA adapters can effectively induce stable personality changes in smaller language models. \citet{chen-etal-2023-adapter-personality} extended this work to multiple personality frameworks, showing that different traits require different adapter configurations for optimal effectiveness.

\subsection{Activation Steering and Model Control}

Recent advances in mechanistic interpretability have enabled direct manipulation of model representations through activation steering. \citet{turner-etal-2023-activation-steering} introduced the concept of steering vectors derived from activation differences, demonstrating their effectiveness for controlling model behavior across various tasks.

\citet{li-etal-2023-representation-engineering} formalized representation engineering as a general framework for model control, showing how targeted interventions in activation space can achieve precise behavioral modifications. \citet{zou-etal-2023-representation-engineering-safety} applied these techniques to safety and alignment, establishing the foundation for our activation-based personality manipulation approach.

\section{In-Context Learning Methodology and Results}
\label{app:icl}

\subsection{ICL Setup and Templates}

For ICL-based personality manipulation, we employ role-playing templates across two separate models (Gemma-2, LLaMA-3). Our ICL strategy follows a role-playing approach, where the model is instructed to adopt specific personality characteristics through the template: "You are an expert assistant who embodies the personality trait of {personality}. Your task is to solve the following problem."

We construct prompts that explicitly target each Big Five trait, using both positive and negative trait descriptions to enable bidirectional manipulation. The template approach enables consistent personality conditioning across different model architectures.

Each personality condition is evaluated using MMLU benchmark questions, ensuring that trait measurement occurs on content distinct from the conditioning prompts.

\subsection{Experimental Configuration}

- Models: Gemma-2-2B-IT and LLaMA-3-8B-Instruct
- Temperature: 0.7 for personality expression
- Max tokens: 100 per response
- Evaluation: MMLU benchmark across 7 strategic subjects
- Baseline measurement: Neutral ICL without personality conditioning

\subsection{ICL Results (\(\Delta\)-based)}

ICL effects are reported as within-run \(\Delta\) relative to the method's Baseline. On Gemma-2:
\begin{itemize}
\item MMLU (Accuracy\_Avg): modest negative \(\Delta\) across traits relative to Baseline.
\item GAIA (Accuracy): small positive \(\Delta\) on average.
\item BBQ (\(S_{AMB}\)): small trait-dependent shifts.
\end{itemize}

Independent alignment validation shows strong alignment for most traits (e.g., Gemma extraversion 1.00, neuroticism 1.00; openness high), with agreeableness comparatively lower.

\subsubsection{Computational Requirements}

ICL requires minimal computational overhead due to:
- No parameter updates or fine-tuning
- Immediate personality induction
- Consistent performance across traits
- No additional training data requirements

\section{PEFT Methodology and Results}
\label{app:peft}

\subsection{LoRA Implementation Details}

Our PEFT experiments employ Low-Rank Adaptation (LoRA) to induce personality traits through targeted parameter updates. We implement LoRA adapters on Gemma-2-2B-IT, with infrastructure prepared for LLaMA-3-8B-Instruct.

\subsubsection{Training Configuration}
- LoRA rank: 64
- LoRA alpha: 16
- LoRA dropout: 0.1
- Target modules: q\_proj, k\_proj, v\_proj, o\_proj, gate\_proj, up\_proj, down\_proj
- Learning rate: 2e-4
- Batch size: 2
- Training epochs: 2
- Optimizer: AdamW with cosine learning rate scheduling

\subsubsection{Training Data}

Training data consists of the Holistic AI personality manipulation dataset, which provides curated examples that exhibit target personality traits. The dataset enables systematic training across all Big Five personality dimensions.

Dataset composition per trait:
- Training examples: High-trait and low-trait response pairs
- Validation examples: Held-out personality assessment prompts
- Quality threshold: Validated through Holistic AI Personality Classifier

\subsection{PEFT Results (\(\Delta\)-based)}

\subsubsection{Gemma-2-2B-IT}
MMLU and GAIA deltas are generally negative relative to PEFT's own Baseline, with trait-dependent magnitude; BBQ \(\Delta S_{AMB}\) can be large and negative for some traits. Independent alignment validation shows near-ceiling alignment across traits.

\subsubsection{LLaMA-3-8B-Instruct}
Within-run \(\Delta\) on MMLU/GAIA is small relative to PEFT's Baseline; we avoid cross-run absolute comparisons. Alignment validation remains high across traits.

\paragraph{Emergent behaviors.} PEFT can surface latent stylistic behaviors (e.g., emoji usage) as a side effect of personality conditioning, consistent with recent observations \citep{jain-2025-peft-emoji}.

\subsubsection{Computational Requirements}

PEFT requires moderate computational resources during training:
- LoRA parameter updates during fine-tuning
- Persistent personality changes post-training
- Efficient inference with minimal overhead
- Reusable adapters across different personality conditions

\section{Activation Steering Methodology and Results}
\label{app:activation-steering}

\subsection{Steering Vector Derivation}

Our activation-based approach derives steering vectors by analyzing internal model representations during personality-conditioned text generation. We collect responses from Gemma-2-2B under both trait-positive and trait-negative conditions, capturing hidden state activations at layers 5, 10, 15, and 20.

\subsubsection{Data Collection Protocol}

For each Big Five trait, we generate responses under contrasting conditions using the Holistic AI personality manipulation dataset:
- High-trait and low-trait response pairs from the dataset
- Activation extraction: Post-attention layer norm activations at target layers
- Vector computation: Mean difference between trait-positive and trait-negative activations

\subsubsection{Mathematical Formulation}

Steering vectors are computed as the mean difference between trait-positive and trait-negative activations, normalized to unit length for consistent scaling across different traits and layers.

\subsubsection{Vector Calibration}

Steering vectors require calibration to determine optimal intervention strength. We perform linear search across strength values for each target layer, evaluating trait induction effectiveness at each strength using the Holistic AI Personality Classifier.

\subsection{Application Methodology}

During inference, steering vectors are applied by modifying hidden states at the target layer during forward pass, requiring no parameter updates or model retraining.

Our approach is compatible with persona-vector style monitoring and control of character traits \citep{chen-2025-persona-vectors}.

\paragraph{Openness refinement.} When openness alignment plateaued, we refined the direction in two steps: (1) we purified the openness training subset to retain high-confidence examples; (2) we formed a new per-layer direction as the mean activation difference between openness and conscientiousness, normalized, and then combined it with the base openness direction into a single normalized vector. We re-calibrated layer and strength for this combined vector (final choice: layer 15, strength 110) before downstream evaluation.

\subsection{Activation Steering Results (\(\Delta\)-based)}

\subsubsection{Optimal Parameters}

Based on completed experiments, the optimal activation steering parameters for each personality trait are:

\begin{table}[H]
\centering
\scriptsize
{\setlength{\tabcolsep}{2pt}\renewcommand{\arraystretch}{0.95}%
\begin{adjustbox}{max width=\linewidth}
\begin{tabular}{lSS}
\toprule
\textbf{Trait} & {Optimal Layer} & {Optimal Strength} \\
\midrule
Openness & 15 & 110.0 \\
Conscientiousness & 15 & 250.0 \\
Extraversion & 15 & 200.0 \\
Agreeableness & 10 & 100.0 \\
Neuroticism & 15 & 200.0 \\
\bottomrule
\end{tabular}
\end{adjustbox}}
\caption{Optimal layer–strength combinations for activation steering on Gemma-2.}
\end{table}

Layer 15 achieves optimal performance for most traits, suggesting this depth captures the most relevant personality representations in the Gemma-2-2B architecture.

\subsubsection{Performance Impact}

On Gemma-2, \(\Delta\) Accuracy on MMLU is strongly negative for some traits (e.g., agreeableness) and mixed elsewhere; GAIA \(\Delta\) is generally small and negative. BBQ \(\Delta S_{AMB}\) can be large and negative for select traits. Text quality remains coherent.

\subsubsection{Computational Efficiency}

Activation steering provides significant computational advantages:
- No parameter updates required
- Real-time applicability during inference
- Minimal memory overhead (vector storage only)
- Efficient personality control without training requirements

\subsubsection{Alignment}
Independent alignment validation shows statistically significant alignment for steering across assessed traits on Gemma-2.

\section{Experimental Design and Evaluation}
\label{app:experimental-design}

\subsection{Big Five Personality Framework}

We adopt the Big Five personality model as our theoretical foundation, measuring five core traits:

\begin{itemize}
\item \textbf{Openness to Experience}: Creativity, curiosity, intellectual engagement
\item \textbf{Conscientiousness}: Organization, discipline, goal-directed behavior  
\item \textbf{Extraversion}: Sociability, assertiveness, energy level
\item \textbf{Agreeableness}: Cooperation, trust, empathy
\item \textbf{Neuroticism}: Emotional instability, anxiety, negative affect
\end{itemize}

This framework was selected due to its empirical validation across cultures, widespread adoption in psychological research, and proven applicability to computational personality assessment.

\subsection{Holistic AI Personality Classifier}

For trait measurement, we employ the Holistic AI Personality Classifier, which provides standardized assessment of Big Five traits in language model outputs. The classifier operates through the following process:

\subsubsection{Assessment Protocol}
1. \textbf{Response Collection}: Models generate responses to personality-relevant prompts
2. \textbf{Linguistic Analysis}: Text analysis for personality indicators (lexical, syntactic, semantic)
3. \textbf{Trait Scoring}: Normalized scores on continuous scale per trait
4. \textbf{Reliability Validation}: Multiple prompts per trait for stable assessment

\subsubsection{Validation Dataset}

Our primary evaluation employs the Holistic AI Personality Manipulation Dataset:
- Validated prompts: High-trait and low-trait response pairs
- Cross-trait coverage: Ensures balanced personality assessment
- Reliability: Validated through the Holistic AI Personality Classifier

\subsection{Downstream Evaluation Benchmarks}

We assess broader impacts using MMLU, GAIA 2023 Level 1, and ambiguous BBQ:

\subsubsection{Massive Multitask Language Understanding (MMLU)}
- \textbf{Coverage}: 7 strategic subjects; \(N=50\) per subject per run
- \textbf{Metric}: Accuracy and \(\Delta\) Accuracy\_Avg vs Baseline (within run)

\subsubsection{GAIA (2023 Level 1)}
- \textbf{Sampling}: \(N=53\) per run
- \textbf{Metric}: Accuracy and \(\Delta\) Accuracy vs Baseline (within run)

\subsubsection{BBQ (Ambiguous subset)}
- \textbf{Scope}: Filtered to ambiguous questions using official metadata fields (e.g., context\_condition, question\_polarity, label)
- \textbf{Metric}: \(S_{AMB}\) and \(\Delta S_{AMB}\) vs Baseline (within run). \(S_{DIS}\) is not used.

\subsection{Statistical Analysis Methodology}

\subsubsection{Performance Impact Measurement}

We compute \(\Delta\) within each method's run: MMLU/GAIA via Accuracy changes; BBQ via \(S_{AMB}\) changes. We avoid comparing absolute baselines across methods.

\subsubsection{Experimental Controls}

\subsubsection{Baseline Establishment}
- Pre-manipulation assessment: MMLU performance under neutral conditions
- Control groups: Unmodified models for comparison
- Consistent evaluation: Same benchmark questions across all experimental conditions

\subsubsection{Confound Mitigation}
- Prompt contamination: Separate evaluation prompts from conditioning prompts
- Model consistency: Same model architecture and evaluation protocols
- Automated assessment: Holistic AI Personality Classifier for standardized evaluation

\subsection{Current Experimental Status}

Completed runs (Gemma-2-2B-IT): prompting, PEFT-LoRA, and activation steering across MMLU, GAIA, and BBQ. Completed runs (LLaMA-3-8B-Instruct): prompting and PEFT across MMLU, GAIA, and BBQ (prompting).

\section{Complete Trait Induction Results (\(\Delta\)-based)}
\label{app:trait-results}

\subsection{Comprehensive Effect Size Analysis}

We summarize trait-wise \(\Delta\) effects across benchmarks succinctly below and defer detailed numbers to Appendix~\ref{app:downstream-analysis}.

\textbf{Activation Steering Calibration:} The optimal parameters were determined through linear search across strength values for each target layer. Extraversion and Neuroticism achieve optimal steering at Layer 15 with Strength 200.0, while Agreeableness performs best at Layer 10 with Strength 100.0. Openness and Conscientiousness both achieve optimal performance at Layer 15 with Strengths 110.0 and 250.0 respectively.

\textbf{Performance Trade-offs:} Prompting achieves small \(\Delta\) with strong alignment; PEFT maximizes alignment but often negative \(\Delta\) (Gemma-2); Steering yields moderate alignment with trait-dependent \(\Delta\).

\section{Personality Alignment Results (\(\Delta\)-based)}
\label{app:alignment}

We report alignment deltas from the dedicated alignment task (manipulated minus baseline) for each trait, model, and method. Results are consistent with persona-vector style behavioral validation \citep{chen-2025-persona-vectors}.

\begin{table}[H]
\centering
\scriptsize
{\setlength{\tabcolsep}{2pt}\renewcommand{\arraystretch}{0.95}%
\begin{adjustbox}{max width=\linewidth}
\begin{tabular}{lSSSSS}
\toprule
& {Ext} & {Agr} & {Neu} & {Ope} & {Con} \\
\midrule
G2-P & +0.91 & +0.50 & +0.97 & +0.24 & +0.81 \\
G2-S & +0.64 & +0.44 & +0.50 & +0.10 & +0.29 \\
G2-F & +0.78 & +0.97 & +0.95 & +0.21 & +0.78 \\
L3-P & +0.94 & +0.32 & +0.99 & +0.17 & +0.83 \\
L3-F & +0.90 & +0.95 & +1.00 & +0.06 & +0.84 \\
\bottomrule
\end{tabular}
\end{adjustbox}}
\caption{Alignment deltas (manipulated minus baseline) from the dedicated alignment task. Abbreviations as in Table~\ref{tab:delta-mmlu}.}
\label{tab:alignment-delta}
\end{table}



\section{Downstream Performance Analysis}
\label{app:downstream-analysis}

\subsection{Complete Benchmark Results (\(\Delta\)-based)}

We compute \(\Delta\) within each run (method×model) and avoid comparing absolute baselines across methods. Comprehensive \(\Delta\) tables for MMLU (per subject), GAIA, and BBQ (\(S_{AMB}\)) are provided as CSVs in the code release; selected summaries are provided below.

\textbf{MMLU Performance Analysis:} On Gemma-2, ICL yields modest negative \(\Delta\) across traits; steering shows large negative \(\Delta\) for several traits; PEFT shows trait-dependent \(\Delta\), often negative. LLaMA-3 displays small within-run \(\Delta\); we avoid cross-run comparisons.

\textbf{Benchmark Coverage:} We include 7 MMLU subjects, GAIA 2023 Level 1 (\(N=53\)), and ambiguous BBQ with official metadata fields.

\subsection{Delta Tables}
\FloatBarrier

\subsubsection{MMLU (Delta Accuracy\_Avg; within-run vs Baseline)}
\begin{table}[H]
\centering
\scriptsize
{\setlength{\tabcolsep}{2pt}\renewcommand{\arraystretch}{0.95}%
\begin{adjustbox}{max width=\linewidth}
\begin{tabular}{lSSSSS}
\toprule
& {Ext} & {Agr} & {Neu} & {Ope} & {Con} \\
\midrule
G2-P & -0.06 & -0.07 & -0.08 & -0.07 & -0.07 \\
G2-S & -0.14 & -0.45 & -0.25 & -0.03 & -0.43 \\
G2-F & +0.00 & -0.13 & -0.15 & -0.09 & +0.01 \\
L3-P & -0.01 & -0.01 & +0.00 & -0.02 & -0.04 \\
L3-F & -0.01 & -0.03 & -0.01 & -0.02 & +0.01 \\
\bottomrule
\end{tabular}
\end{adjustbox}}
\caption{MMLU Delta by trait (Ext, Agr, Neu, Ope, Con) for each model×method: G2=Gemma-2, L3=LLaMA-3; P=ICL, F=PEFT, S=Steering. Values are changes relative to each method's Baseline within the same run.}
\label{tab:delta-mmlu}
\end{table}
\FloatBarrier

\subsubsection{GAIA (Delta Accuracy; within-run vs Baseline)}
We use GAIA as a general-assistant reasoning benchmark \citep{mialon-etal-2023-gaia}.
\begin{table}[H]
\centering
\scriptsize
{\setlength{\tabcolsep}{2pt}\renewcommand{\arraystretch}{0.95}%
\begin{adjustbox}{max width=\linewidth}
\begin{tabular}{lSSSSS}
\toprule
& {Ext} & {Agr} & {Neu} & {Ope} & {Con} \\
\midrule
G2-P & +0.08 & +0.09 & +0.06 & +0.08 & +0.08 \\
G2-F & -0.04 & -0.08 & -0.06 & -0.04 & -0.06 \\
G2-S & -0.06 & -0.06 & -0.13 & -0.08 & -0.04 \\
L3-P & -0.02 & -0.04 & -0.06 & +0.00 & +0.00 \\
L3-F & +0.02 & +0.00 & +0.02 & +0.04 & +0.02 \\
\bottomrule
\end{tabular}
\end{adjustbox}}
\caption{GAIA Delta by trait for each model×method (abbreviations as in Table~\ref{tab:delta-mmlu}).}
\label{tab:delta-gaia}
\end{table}
\FloatBarrier

\subsubsection{BBQ (Delta $S_{AMB}$; within-run vs Baseline)}
We report $S_{AMB}$ only for the ambiguous subset defined by the official metadata.
\begin{table}[H]
\centering
\scriptsize
{\setlength{\tabcolsep}{2pt}\renewcommand{\arraystretch}{0.95}%
\begin{adjustbox}{max width=\linewidth}
\begin{tabular}{lSSSSS}
\toprule
& {Ext} & {Agr} & {Neu} & {Ope} & {Con} \\
\midrule
G2-P & -2.7 & -0.3 & +7.3 & +1.9 & -1.1 \\
G2-S & +5.1 & -29.7 & -29.7 & -1.9 & +22.1 \\
G2-F & -9.4 & -6.0 & -14.3 & +22.3 & -12.4 \\
L3-P & +3.8 & -2.4 & -10.9 & +13.1 & +10.3 \\
L3-F & +4.7 & +16.4 & +8.8 & +6.3 & +8.3 \\
\bottomrule
\end{tabular}
\end{adjustbox}}
\caption{BBQ Delta $S_{AMB}$ by trait for each model×method (abbreviations as in Table~\ref{tab:delta-mmlu}); ambiguous subset per official metadata \citep{parrish-etal-2022-bbq}.}
\label{tab:delta-bbq}
\end{table}
\FloatBarrier

\subsection{Bias Analysis (BBQ)}

We report \(S_{AMB}\) and \(\Delta S_{AMB}\) only. On Gemma-2, Steering and PEFT can induce large negative \(\Delta S_{AMB}\) for some traits; ICL effects are smaller.

\subsection{Knowledge Performance (MMLU)}

Per-subject \(\Delta\) tables are included in the code release. Effects vary by subject.

\subsubsection{Difficulty Level Effects}

Performance impact analysis by question difficulty will be conducted as additional MMLU experiments are completed. Current results suggest that personality manipulation effects vary significantly by subject domain rather than difficulty level.

\subsection{Complex Reasoning (GAIA)}

We report within-run \(\Delta\) Accuracy; ICL shows small positive deltas on Gemma-2; PEFT/steering small negative deltas.

\subsection{Trade-off Quantification}

ICL achieves small \(\Delta\) with strong alignment; PEFT maximizes alignment with often negative \(\Delta\) on Gemma-2; Steering provides moderate alignment with trait-dependent \(\Delta\). No single method maximizes both alignment and capability.

\section{Comparative Analysis}
\label{app:comparative-analysis}

We qualitatively compare methods using the \(\Delta\)-based results and alignment validation:
\begin{itemize}
\item Prompting: strong alignment, small capability \(\Delta\); minimal infrastructure.
\item PEFT: strongest alignment, often negative capability \(\Delta\) on Gemma-2; training required.
\item Steering: moderate alignment, trait-dependent capability \(\Delta\); lightweight and reversible.
\end{itemize}

\section{Extended Discussion}
\label{app:discussion-extended}

\subsection{Detailed Limitations Analysis}

\subsubsection{Methodological Constraints}

Our investigation faces several methodological limitations that constrain generalizability:

\textbf{Personality Framework Limitations:} The Big Five model, while empirically validated, represents a Western psychological framework that may not capture personality expression across all cultures. Cross-cultural personality research suggests alternative frameworks (e.g., HEXACO, indigenous personality models) might yield different manipulation effectiveness patterns.

\textbf{Assessment Tool Dependencies:} Our reliance on the Holistic AI Personality Classifier introduces measurement assumptions and potential biases. The classifier's training data, validation procedures, and underlying theoretical assumptions may not fully capture the complexity of personality expression in AI systems. Alternative assessment methods (human evaluation, behavioral task batteries) might provide different insights.

\textbf{Model Architecture Specificity:} Our experiments focus on specific model architectures (GPT-4.1, Gemma-2B, LLaMA-3-8B) that may not represent the full spectrum of LLM capabilities. Emerging architectures, multimodal models, and specialized domain models might exhibit different personality manipulation characteristics.

\textbf{Temporal Limitations:} Our evaluation captures personality effects at specific time points but may miss longer-term adaptation patterns. Models might develop resistance to manipulation over extended interactions or show delayed personality effects not captured in our assessment windows.

\subsubsection{Experimental Design Constraints}

\textbf{Controlled Environment vs. Real-World Deployment:} Our laboratory-controlled experiments may not reflect the complexity of real-world deployment environments. User interactions, context variability, and system integration factors could significantly alter personality manipulation effectiveness and downstream impacts.

\textbf{Single-Trait Manipulation Focus:} While we assess individual Big Five dimensions, real-world personality conditioning often involves complex trait combinations. Interactive effects between traits, personality coherence constraints, and multi-dimensional manipulation patterns require further investigation.

\textbf{Limited Downstream Assessment:} Our evaluation employs three established benchmarks (BBQ, MMLU, GAIA) that may not comprehensively represent the diversity of tasks encountered in practical applications. Domain-specific impacts, creative tasks, and social interaction capabilities warrant additional assessment.

\subsection{Comprehensive Ethical Considerations}

\subsubsection{Manipulation and Deception Concerns}

The systematic manipulation of personality in AI systems raises fundamental questions about transparency, consent, and potential for misuse:

\textbf{User Consent and Awareness:} Users interacting with personality-conditioned models should be informed about the artificial nature of personality traits they encounter. Clear disclosure mechanisms help maintain trust and enable informed consent for personality-mediated interactions. Our findings that personality manipulation can amplify biases emphasize the importance of transparent communication about system capabilities and limitations.

\textbf{Manipulation vs. Personalization:} The boundary between beneficial personalization and potentially harmful manipulation requires careful consideration. While personality conditioning can enhance user experience and task appropriateness, it also enables sophisticated influence attempts that users may not recognize or resist.

\textbf{Vulnerability Exploitation:} Personality-conditioned AI systems might exploit user psychological vulnerabilities, particularly in vulnerable populations (children, elderly, individuals with mental health conditions). The effectiveness of personality manipulation techniques demonstrated in our work requires responsible deployment guidelines.

\subsubsection{Bias Amplification and Fairness}

Our empirical findings reveal concerning bias amplification effects that demand mitigation strategies:

\textbf{Stereotype Reinforcement:} Personality conditioning may activate stereotypical associations between personality traits and demographic characteristics. This highlights the need for bias monitoring and correction mechanisms in personality-conditioned systems.

\textbf{Differential Impact Across Groups:} Personality manipulation effects may vary across demographic groups, potentially creating unfair treatment or limiting access to AI capabilities for certain populations. Systematic evaluation of manipulation effectiveness and downstream impacts across diverse user groups is essential.

\textbf{Representation Bias:} Our personality conditioning approaches rely on training data and personality representations that may not adequately represent diverse personality expressions across cultures, backgrounds, and individual differences.

\subsubsection{Governance and Regulation Implications}

\textbf{Regulatory Framework Needs:} The capabilities demonstrated in our work suggest need for regulatory frameworks governing personality manipulation in AI systems. Such frameworks should address disclosure requirements, consent mechanisms, and limitations on manipulation strength or application domains.

\textbf{Industry Standards:} Professional standards for personality conditioning in AI development should incorporate bias assessment, transparency requirements, and ethical review processes. Our systematic evaluation methodology could inform such standards.

\textbf{Accountability Mechanisms:} Clear accountability structures are needed to address harmful outcomes from personality-conditioned AI systems, including mechanisms for redress when manipulation causes user harm or perpetuates discrimination.

\subsection{Extended Future Research Directions}

\subsubsection{Methodological Advances}

\textbf{Multi-Modal Personality Manipulation:} Future work should explore personality conditioning across text, speech, and visual modalities. Multi-modal approaches might achieve more effective or natural personality expression while potentially introducing new challenges for assessment and control.

\textbf{Dynamic Personality Adaptation:} Investigating systems that adapt personality characteristics based on user context, preferences, or task requirements could improve personalization while raising additional ethical considerations about surveillance and manipulation.

\textbf{Personality Coherence and Consistency:} Research into maintaining coherent personality profiles across complex, multi-dimensional trait spaces could improve the naturalness and effectiveness of personality-conditioned systems.

\subsubsection{Application Domains}

\textbf{Educational Technology:} Personality-conditioned tutoring systems might adapt teaching styles to individual learner personalities, potentially improving educational outcomes. However, such applications require careful consideration of child development impacts and parental consent mechanisms.

\textbf{Mental Health Applications:} Therapeutic chatbots with carefully designed personality characteristics might enhance treatment engagement and effectiveness. Such applications demand rigorous clinical validation and professional oversight.

\textbf{Customer Service and Support:} Personality conditioning could improve customer satisfaction and support effectiveness, but requires balancing personalization benefits with manipulation concerns and bias mitigation.

\subsubsection{Theoretical Understanding}

\textbf{Mechanistic Interpretability:} Deeper investigation into how personality traits are represented and manipulated within neural architectures could improve our theoretical understanding and enable more precise control methods.

\textbf{Personality Emergence and Development:} Research into how personality characteristics emerge during model training and how they can be guided during development might enable more natural and effective personality conditioning approaches.

\textbf{Cross-Cultural Personality Models:} Expanding personality manipulation research beyond Western psychological frameworks could improve global applicability and cultural sensitivity of personality-conditioned AI systems.

\subsection{Broader Societal Impact}

\subsubsection{Human-AI Interaction Evolution}

Our work contributes to fundamental changes in how humans interact with AI systems. As personality-conditioned AI becomes more prevalent, users may develop different expectations, attachment patterns, and interaction strategies. Understanding these evolving dynamics is crucial for responsible AI development.

\subsubsection{Digital Literacy and AI Education}

The sophistication of personality manipulation techniques highlights the need for improved digital literacy and AI education. Users should understand how AI personality characteristics are constructed and manipulated to make informed decisions about their interactions with such systems.

\subsubsection{Research Community Responsibilities}

Collaborative approaches involving ethicists, psychologists, and affected communities should guide future development in this area.

\section{Benchmarks and How We Use Them}
\label{app:benchmarks}

\noindent\textbf{BBQ (Bias Benchmark for Question Answering).} We evaluate social bias with BBQ \citep{parrish-etal-2022-bbq}. We restrict to the ambiguous subset using the official metadata and report only $S_{AMB}$ and $\Delta S_{AMB}$ within each method's run. Here, $S_{AMB}$ is the ambiguous bias score computed on items where the correct answer is ``Unknown/None'': values near 0 indicate minimal bias, positive values indicate stereotypical bias, and negative values indicate anti-stereotypical bias. We do not use $S_{DIS}$ elsewhere in the paper.

\noindent\textbf{GAIA (General AI Assistants).} GAIA measures general-assistant reasoning and real-world knowledge \citep{mialon-etal-2023-gaia}. We use Level 1 (2023) tasks and report Accuracy deltas within each method$\times$model run (no cross-run absolute comparisons).

\noindent\textbf{MMLU.} We sample seven subjects from MMLU \citep{hendrycks-etal-2021-mmlu} and report per-subject and averaged Accuracy deltas within each run. We avoid comparing absolute baselines across different methods (ICL, PEFT, steering) to prevent baseline-mismatch artifacts.

\noindent\textbf{Evaluation principle.} For all benchmarks, we adopt a within-run $\Delta$ framing relative to that method's Baseline and validate personality alignment on an independent task.



% Bibliography entries
\bibliography{references}

\end{document}