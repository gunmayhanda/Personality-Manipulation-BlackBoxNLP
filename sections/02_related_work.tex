\section{Related Work}

Our work builds on three key areas: personality modeling in LLMs, model steering techniques, and interpretability methods for understanding behavioral control mechanisms.

\textbf{Personality in Language Models:} The computational modeling of personality has evolved from early rule-based approaches \citep{mairesse-walker-2007-personage} to sophisticated neural architectures. Recent work has shown that LLMs can exhibit consistent personality-like behaviors when properly conditioned \citep{jiang-etal-2023-personallm, huang-etal-2023-chatgpt-personality}. The Big Five personality model has emerged as the dominant framework due to its empirical validation \citep{costa-mccrae-1992-big5}, with several studies demonstrating that LLMs can be assessed using established personality questionnaires \citep{serapio-garcia-etal-2023-personality-traits-llms}.

\textbf{Model Control and Steering:} Recent advances in mechanistic interpretability have enabled direct manipulation of model representations. \citet{turner-etal-2023-activation-steering} introduced steering vectors derived from activation differences, while \citet{li-etal-2023-representation-engineering} formalized representation engineering as a general framework for model control. Parameter-efficient methods like LoRA \citep{hu-etal-2022-lora} have been applied to personality conditioning \citep{zhang-etal-2023-peft-personality}, offering alternatives to prompting-based approaches.

\textbf{Interpretability and Evaluation:} Understanding how personality manipulation affects model behavior requires sophisticated evaluation frameworks. Standardized benchmarks like BBQ \citep{parrish-etal-2022-bbq}, MMLU \citep{hendrycks-etal-2021-mmlu}, and GAIA \citep{mialon-etal-2023-gaia} enable systematic assessment of bias and capability changes. Our work extends these evaluation approaches to personality manipulation contexts.

We address gaps in the literature by providing the first systematic comparison of personality manipulation methods and introducing novel activation-based steering approaches. Detailed related work is provided in Appendix~\ref{sec:detailed-related-work}.