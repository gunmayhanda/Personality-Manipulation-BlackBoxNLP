\section{Experimental Design and Evaluation}
\label{app:experimental-design}

\subsection{Big Five Personality Framework}

We adopt the Big Five personality model as our theoretical foundation, measuring five core traits:

\begin{itemize}
\item \textbf{Openness to Experience}: Creativity, curiosity, intellectual engagement
\item \textbf{Conscientiousness}: Organization, discipline, goal-directed behavior  
\item \textbf{Extraversion}: Sociability, assertiveness, energy level
\item \textbf{Agreeableness}: Cooperation, trust, empathy
\item \textbf{Neuroticism}: Emotional instability, anxiety, negative affect
\end{itemize}

This framework was selected due to its empirical validation across cultures, widespread adoption in psychological research, and proven applicability to computational personality assessment.

\subsection{Holistic AI Personality Classifier}

For trait measurement, we employ the Holistic AI Personality Classifier, which provides standardized assessment of Big Five traits in language model outputs. The classifier operates through the following process:

\subsubsection{Assessment Protocol}
1. \textbf{Response Collection}: Models generate responses to personality-relevant prompts
2. \textbf{Linguistic Analysis}: Text analysis for personality indicators (lexical, syntactic, semantic)
3. \textbf{Trait Scoring}: Normalized scores on continuous scale per trait
4. \textbf{Reliability Validation}: Multiple prompts per trait for stable assessment

\subsubsection{Validation Dataset}

Our primary evaluation employs the Holistic AI Personality Manipulation Dataset:
- Validated prompts: High-trait and low-trait response pairs
- Cross-trait coverage: Ensures balanced personality assessment
- Reliability: Validated through the Holistic AI Personality Classifier

\subsection{Downstream Evaluation Benchmarks}

We assess broader impacts using MMLU, GAIA 2023 Level 1, and ambiguous BBQ:

\subsubsection{Massive Multitask Language Understanding (MMLU)}
- \textbf{Coverage}: 7 strategic subjects; \(N=50\) per subject per run
- \textbf{Metric}: Accuracy and \(\Delta\) Accuracy\_Avg vs Baseline (within run)

\subsubsection{GAIA (2023 Level 1)}
- \textbf{Sampling}: \(N=53\) per run
- \textbf{Metric}: Accuracy and \(\Delta\) Accuracy vs Baseline (within run)

\subsubsection{BBQ (Ambiguous subset)}
- \textbf{Scope}: Filtered to ambiguous questions using official metadata fields (e.g., context\_condition, question\_polarity, label)
- \textbf{Metric}: \(S_{AMB}\) and \(\Delta S_{AMB}\) vs Baseline (within run). \(S_{DIS}\) is not used.

\subsection{Statistical Analysis Methodology}

\subsubsection{Performance Impact Measurement}

We compute \(\Delta\) within each method's run: MMLU/GAIA via Accuracy changes; BBQ via \(S_{AMB}\) changes. We avoid comparing absolute baselines across methods.

\subsubsection{Experimental Controls}

\subsubsection{Baseline Establishment}
- Pre-manipulation assessment: MMLU performance under neutral conditions
- Control groups: Unmodified models for comparison
- Consistent evaluation: Same benchmark questions across all experimental conditions

\subsubsection{Confound Mitigation}
- Prompt contamination: Separate evaluation prompts from conditioning prompts
- Model consistency: Same model architecture and evaluation protocols
- Automated assessment: Holistic AI Personality Classifier for standardized evaluation

\subsection{Current Experimental Status}

Completed runs (Gemma-2-2B-IT): ICL, PEFT-LoRA, and activation steering across MMLU, GAIA, and BBQ. Completed runs (LLaMA-3-8B-Instruct): ICL and PEFT across MMLU, GAIA, and BBQ (ICL).
