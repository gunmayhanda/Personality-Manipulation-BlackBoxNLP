\section{Extended Discussion}
\label{app:discussion-extended}

\subsection{Detailed Limitations Analysis}

\subsubsection{Methodological Constraints}

Our investigation faces several methodological limitations that constrain generalizability:

\textbf{Personality Framework Limitations:} The Big Five model, while empirically validated, represents a Western psychological framework that may not capture personality expression across all cultures. Cross-cultural personality research suggests alternative frameworks (e.g., HEXACO, indigenous personality models) might yield different manipulation effectiveness patterns.

\textbf{Assessment Tool Dependencies:} Our reliance on the Holistic AI Personality Classifier introduces measurement assumptions and potential biases. The classifier's training data, validation procedures, and underlying theoretical assumptions may not fully capture the complexity of personality expression in AI systems. Alternative assessment methods (human evaluation, behavioral task batteries) might provide different insights.

\textbf{Model Architecture Specificity:} Our experiments focus on specific model architectures (GPT-4.1, Gemma-2B, LLaMA-3-8B) that may not represent the full spectrum of LLM capabilities. Emerging architectures, multimodal models, and specialized domain models might exhibit different personality manipulation characteristics.

\textbf{Temporal Limitations:} Our evaluation captures personality effects at specific time points but may miss longer-term adaptation patterns. Models might develop resistance to manipulation over extended interactions or show delayed personality effects not captured in our assessment windows.

\subsubsection{Experimental Design Constraints}

\textbf{Controlled Environment vs. Real-World Deployment:} Our laboratory-controlled experiments may not reflect the complexity of real-world deployment environments. User interactions, context variability, and system integration factors could significantly alter personality manipulation effectiveness and downstream impacts.

\textbf{Single-Trait Manipulation Focus:} While we assess individual Big Five dimensions, real-world personality conditioning often involves complex trait combinations. Interactive effects between traits, personality coherence constraints, and multi-dimensional manipulation patterns require further investigation.

\textbf{Limited Downstream Assessment:} Our evaluation employs three established benchmarks (BBQ, MMLU, GAIA) that may not comprehensively represent the diversity of tasks encountered in practical applications. Domain-specific impacts, creative tasks, and social interaction capabilities warrant additional assessment.

\subsection{Comprehensive Ethical Considerations}

\subsubsection{Manipulation and Deception Concerns}

The systematic manipulation of personality in AI systems raises fundamental questions about transparency, consent, and potential for misuse:

\textbf{User Consent and Awareness:} Users interacting with personality-conditioned models should be informed about the artificial nature of personality traits they encounter. Clear disclosure mechanisms help maintain trust and enable informed consent for personality-mediated interactions. Our findings that personality manipulation can amplify biases emphasize the importance of transparent communication about system capabilities and limitations.

\textbf{Manipulation vs. Personalization:} The boundary between beneficial personalization and potentially harmful manipulation requires careful consideration. While personality conditioning can enhance user experience and task appropriateness, it also enables sophisticated influence attempts that users may not recognize or resist.

\textbf{Vulnerability Exploitation:} Personality-conditioned AI systems might exploit user psychological vulnerabilities, particularly in vulnerable populations (children, elderly, individuals with mental health conditions). The effectiveness of personality manipulation techniques demonstrated in our work requires responsible deployment guidelines.

\subsubsection{Bias Amplification and Fairness}

Our empirical findings reveal concerning bias amplification effects that demand mitigation strategies:

\textbf{Stereotype Reinforcement:} Personality conditioning may activate stereotypical associations between personality traits and demographic characteristics. This highlights the need for bias monitoring and correction mechanisms in personality-conditioned systems.

\textbf{Differential Impact Across Groups:} Personality manipulation effects may vary across demographic groups, potentially creating unfair treatment or limiting access to AI capabilities for certain populations. Systematic evaluation of manipulation effectiveness and downstream impacts across diverse user groups is essential.

\textbf{Representation Bias:} Our personality conditioning approaches rely on training data and personality representations that may not adequately represent diverse personality expressions across cultures, backgrounds, and individual differences.

\subsubsection{Governance and Regulation Implications}

\textbf{Regulatory Framework Needs:} The capabilities demonstrated in our work suggest need for regulatory frameworks governing personality manipulation in AI systems. Such frameworks should address disclosure requirements, consent mechanisms, and limitations on manipulation strength or application domains.

\textbf{Industry Standards:} Professional standards for personality conditioning in AI development should incorporate bias assessment, transparency requirements, and ethical review processes. Our systematic evaluation methodology could inform such standards.

\textbf{Accountability Mechanisms:} Clear accountability structures are needed to address harmful outcomes from personality-conditioned AI systems, including mechanisms for redress when manipulation causes user harm or perpetuates discrimination.

\subsection{Extended Future Research Directions}

\subsubsection{Methodological Advances}

\textbf{Multi-Modal Personality Manipulation:} Future work should explore personality conditioning across text, speech, and visual modalities. Multi-modal approaches might achieve more effective or natural personality expression while potentially introducing new challenges for assessment and control.

\textbf{Dynamic Personality Adaptation:} Investigating systems that adapt personality characteristics based on user context, preferences, or task requirements could improve personalization while raising additional ethical considerations about surveillance and manipulation.

\textbf{Personality Coherence and Consistency:} Research into maintaining coherent personality profiles across complex, multi-dimensional trait spaces could improve the naturalness and effectiveness of personality-conditioned systems.

\subsubsection{Application Domains}

\textbf{Educational Technology:} Personality-conditioned tutoring systems might adapt teaching styles to individual learner personalities, potentially improving educational outcomes. However, such applications require careful consideration of child development impacts and parental consent mechanisms.

\textbf{Mental Health Applications:} Therapeutic chatbots with carefully designed personality characteristics might enhance treatment engagement and effectiveness. Such applications demand rigorous clinical validation and professional oversight.

\textbf{Customer Service and Support:} Personality conditioning could improve customer satisfaction and support effectiveness, but requires balancing personalization benefits with manipulation concerns and bias mitigation.

\subsubsection{Theoretical Understanding}

\textbf{Mechanistic Interpretability:} Deeper investigation into how personality traits are represented and manipulated within neural architectures could improve our theoretical understanding and enable more precise control methods.

\textbf{Personality Emergence and Development:} Research into how personality characteristics emerge during model training and how they can be guided during development might enable more natural and effective personality conditioning approaches.

\textbf{Cross-Cultural Personality Models:} Expanding personality manipulation research beyond Western psychological frameworks could improve global applicability and cultural sensitivity of personality-conditioned AI systems.

\subsection{Broader Societal Impact}

\subsubsection{Human-AI Interaction Evolution}

Our work contributes to fundamental changes in how humans interact with AI systems. As personality-conditioned AI becomes more prevalent, users may develop different expectations, attachment patterns, and interaction strategies. Understanding these evolving dynamics is crucial for responsible AI development.

\subsubsection{Digital Literacy and AI Education}

The sophistication of personality manipulation techniques highlights the need for improved digital literacy and AI education. Users should understand how AI personality characteristics are constructed and manipulated to make informed decisions about their interactions with such systems.

\subsubsection{Research Community Responsibilities}

Collaborative approaches involving ethicists, psychologists, and affected communities should guide future development in this area.
